\section{Введение}

Метод Монте-Карло в общем случае применяется для приближенного вычисления интегралов (фактически) произвольной сложности. Суть метода
заключается в том, чтобы оценить плотность распределения, график которой в точности равен подинтегральной функции. Таким образом, 
метод Монте-Карло в некотором роде коррелирует с методом \emph{выборки с отклонением} (\emph{rejective sampling}). 

Мы здесь не будем вдаваться в детали данного метода, опишем лишь алгоритм оценки меры произвольной фигуры методом Монте-Карло:
\begin{enumerate}
    \item Фигура, меру которой необходимо оценить, заключается в другую фигуру, которая:
        \begin{itemize}
            \item <<Достаточно>> оптимальна с точки зрения минимальности замыкающей фигуры.
            \item Может быть <<легко>> использована в качестве области определения для генератора равномерно распределнных точек.
        \end{itemize}  
    \item В замыкающей фигуре генерирутеся <<достаточно>> большое число $n$ равномерно распределенных точек. Для каждой из точки 
        определяется попала ли она в фигуру, площадь которой необходимо вычислить.
    \item Производится подсчет количества точек ($m$), которые попали в фигуру, площадь которой необходимо вычислить. Тогда
        искомая величина может быть оценена по следующей формуле:
        \begin{equation*}
            S * \frac{m}{n},
        \end{equation*}
        где $S$ --- площадь замыкающей фигуры.
\end{enumerate}
В дополнение к описанному алгоритму заметим, что точность метода Монте-Карло является величиной порядка $\frac{1}{n^2}$, что в свою 
очередь делает $n$ <<достаточно>> хорошим при значениях порядка $10^3$. Понятно, что в общем случае (например, если оцениваемая фигура 
сильно разряжена) может потребоваться и гораздо большее количество экспериментов.