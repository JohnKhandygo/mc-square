\section{Постановка задачи}

В рамках данной работы необходимо решить задачу вычисления площади фигуры, состоящей из произвольного набора окружностей
(возможно пересекающихся) методом \emph{Монте-Карло} (\emph{Monte-Carclo}). Решение задачи должно быть представлено в виде программы на 
языке \texttt{C++} или \texttt{Java}. При этом реализация программы должна предусматривать использование соедующих подходов:
\begin{itemize}
    \item Последовательный.
    \item Параллельный, с использованием стандартных средств управления потоками выбранного языка программирования.
    \item Параллельный, с использованием парадигмы \texttt{OpenMP} или \texttt{MPI}.
\end{itemize}

Результаты работы необходимо представить в виде статистики замеров времени исполнения различных решений и их сравнения. Также 
необходимо провести комплексное тестирование разработанных програм для доказания корректности их работы.