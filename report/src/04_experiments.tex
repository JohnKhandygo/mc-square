\section{Эксперименты}

\subsection{Условия проведения экспериментов}

Для реализации поставленной задачи был выбран язык программирования \texttt{Java} версии $8$. Все эксперименты проводились на 
вычислительной машине со следующими характеристиками:
\begin{itemize}
    \item Операционная система Windows 7 Enterprise 64-bit.
    \item Процессор Intel(R) Core(TM) i5 M 520 с двумя физическими ядрами общей частотой $2.4$ гигагерца.
    \item Объем оперативной памяти $8$ гигабайт.
\end{itemize}

Для запуска экспериментов тестирующая программа собиралась в единый \texttt{.jar} файл, включающий в себя все необходимые 
зависимости. Для гарантирования идентичности условий проведения экспериментов все эксперименты могут быть запущены с помощью 
специальных скриптов.

\subsection{Методика проведения экспериментов}

Эксперименты проводились в два этапа:
\begin{enumerate}
    \item Генерация исходных данных (единожды для запуска трех различных решений).
    \item Запуск каждого вида решения по числу различных экземпляров сгенерированных данных и сбор статистики. 
\end{enumerate}
При этом
\begin{itemize}
    \item Исходные данные представляются в виде файлов в формате \texttt{.json}, которые описывают входные данные в стиле POJO 
        (Plain Old Java Objects).
    \item Каждый файл входных данных описывает фигуру, состоящую из $1000$ произвольных окружностей.
    \item Запуск решений, предполагающих использование многопоточности, производится для количества потоков от $1$го до $8$ми.
    \item Замеры времени производятся изнутри самой программы и включают в себя оценку только лишь процесса вычисления.
\end{itemize}

Ниже представлены параметры установки, верные для всех трех решений. 
\begin{center}
    \begin{tabular}{r|l}
        \textbf{Параметр} & \textbf{Значение} \\ \hline
        \textbf{Количество запусков} & 1000 \\ 
        \textbf{Количество генерируемых точек} & 100000
    \end{tabular}
\end{center}

\subsection{Результаты экспериментов}

Ниже представлены результаты измерений для последовательной программы.
\begin{center}
    \begin{tabular}{c|r|c}
        \multicolumn{2}{c|}{\textbf{Количество потоков}} & $1$ \\ \hline \hline
        \multirow{4}{*}{\textbf{Статистика (мс)}} & Минимум & $27$ \\
        & Среднее & $29.54$ \\
        & Дисперсия & $2.74$ \\
        & Максимум & $99$
    \end{tabular}
\end{center}

Ниже представлены результаты измерений для многопоточной программы с использованием стандартных средств языка \texttt{Java}.
\begin{center}
    \begin{tabular}{c|r|c|c|c|c|c|c|c|c}
        \multicolumn{2}{c|}{\textbf{Количество потоков}} & $1$ & $2$ & $3$ & $4$ & $5$ & $6$ & $7$ & $8$ \\ \hline \hline
        \multirow{4}{*}{\textbf{Статистика (мс)}} & Минимум & $28$ & $14$ & $16$ & $14$ & $18$ & $17$ & $16$ & $16$ \\
        & Среднее & $31.47$ & $19.08$ & $20.11$ & $18.09$ & $23.00$ & $20.95$ & $20.33$ & $19.14$ \\
        & Дисперсия & $3.43$ & $5.42$ & $4.60$ & $5.29$ & $5.17$ & $5.11$ & $5.53$ & $8.03$ \\
        & Максимум & $122$ & $117$ & $143$ & $154$ & $164$ & $158$ & $170$ & $253$
    \end{tabular}
\end{center}

Ниже представлены результаты измерений для многопоточной программы с использованием парадигмы \texttt{MPI} для \texttt{Java}.
\begin{center}
    \begin{tabular}{c|r|c|c|c|c|c|c|c|c}
        \multicolumn{2}{c|}{\textbf{Количество потоков}} & $1$ & $2$ & $3$ & $4$ & $5$ & $6$ & $7$ & $8$ \\ \hline \hline
        \multirow{4}{*}{\textbf{Статистика (мс)}} & Минимум & $30$ & $15$ & $10$ & $8$ & $7$ & $6$ & $5$ & $5$ \\
        & Среднее & $33.49$ & $17.75$ & $16.76$ & $15.70$ & $17.59$ & $19.41$ & $19.88$ & $19.20$ \\
        & Дисперсия & $2.99$ & $3.23$ & $4.40$ & $8.66$ & $18.86$ & $30.72$ & $24.53$ & $26.71$ \\
        & Максимум & $89$ & $50$ & $61$ & $255$ & $409$ & $755$ & $155$ & $169$
    \end{tabular}
\end{center}