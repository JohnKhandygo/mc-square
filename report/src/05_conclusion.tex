\section{Вывод}

В рамках данной работы была разработана программа, позволяющая решить задачу приближенного вычисления площади фигуры, состоящей из
произвольного количества окружностей методом Монте-Карло. Программа была реализована с использованием трех различных подходов:
последовательного, параллельного, с использованием стандартных средств обеспечения параллелизма в языке программирования \texttt{Java},
и с использованием реализации \texttt{MPI} для \texttt{Java}. Также были проведены эксперименты для замера времени исполнения 
различных решений. Из результатов экспериментов можно сделать следующие выводы:
\begin{itemize}
    \item Время работы последовательной программы сопоставимо с временем работы обоих параллельных решений, что может 
        свидетельствовать о корректности постановки экспериментов.
    \item Минимальное время работы в случае параллельных программ с использованием двух потоков, в два раза ниже минимального времени 
        работы соответствующих программ в один поток. Данный результат полностью соотносится с тем фактом, что для экспериментов
        была использована машина с двумя физическими ядрами.
    \item При наращивании количества потоков в параллельных версиях программы, начиная с трех потоков, не наблюдается заметного 
        ускорения по сравнению со случаем двух потоков, что также соотносится с окружением, в котором проводились эксперименты.
    \item Минимальное время работы при работе с \texttt{MPI} ниже, чем минимальное время работы при использовании стандартных 
        потоков языка программирования \texttt{Java}, при использовании одинакового количества потоков. В то же время, дисперсия
        времени исполнения в случае работы с \texttt{MPI} выше по сравнению с реализацией на <<чистой>> \texttt{Java} в аналогичных
        условиях. Здесь можно сделать вывод, что нативная реализация \texttt{MPI} в некоторых случаях позволяет выполнить задачу
        более эффективо, однако, в случае пиковых нагрузок, будет демонстрировать показатели, сопоставимые с показателями стандартных
        \texttt{Java}-потоков или хуже (см случай использования $5$и и $6$и потоков).
\end{itemize}
